%% using aastex version 6.3
\documentclass[onecolumn]{aastex63}
% \usepackage[papersize={10in,7in}, left=0.1in,right=0.1in,top=1in,bottom=0.1in]{geometry}
\usepackage[papersize={7.85in,5.0in}, top=2.0cm, bottom=-0.5cm, left=1.0cm, right=0.25cm]{geometry}
\usepackage{graphicx}
\usepackage{ascii}
\usepackage{amsmath}
\usepackage{multirow}
\usepackage{makecell}
\usepackage{caption}
\renewcommand{\familydefault}{\sfdefault}
% \newcommand{\sup}[1]{\textsuperscript{\textbf{}}}
\captionsetup{font=sf}

\begin{document}

\vspace{3cm}
\begin{table*}
    \centering
    \caption{The model grids available with this version. Shown is the name, size, atmospheric model chemical type of either oxygen (O) or carbon (C), the atmospheric model, and a brief description. \vspace{0.2cm}}
    % \setlength{\tabcolsep}{0.76em}
    \begin{tabular}{ l r c c c c c}
    \hline
    \textbf{Grid name} & \textbf{Size} & \textbf{Type} & \textbf{Atmospheric model} & \textbf{Optical constants} & \textbf{References} \\
    \Xhline{3\arrayrulewidth} \\
    \vspace{0.1cm}
    Oss-Orich-aringer & 2,000 & O & COMARCS & Warm silicates & 1, 6 \\
    \vspace{0.1cm}
    Oss-Orich-bb & 2,000 & O & Black body (BB) & Warm silicates & 6\\
    \vspace{0.1cm}
    Crystalline-20-bb & 2,000 & O & BB & 80\% warm silicates, 20\% crystalline silicates & 4, 6\\
    \vspace{0.1cm}
    corundum-20-bb & 2,000 & O & BB & 80\% warm silicates, 20\% corundum silicates &  2, 6\\
    \vspace{0.1cm}
    big-grain & 2,000 & O & BB & Warm silicates with higher maximum dust grain size of 0.35 &  6\\
    \vspace{0.1cm}
    fifth-iron & 500 & O & BB & 80\% warm silicates, 20\% iron grains &  3, 6\\
    \vspace{0.1cm}
    half-iron & 500 & O & BB & 50\% warm silicates, 50\% iron grains &  3, 6 \\
    \vspace{0.1cm}
    one-fifth-carbon & 500 & O & BB & 80\% warm silicates, 20\% carbonaceous grains &  6, 7\\
    \vspace{0.1cm}
    arnold-palmer & 500 & O & BB & 50\% warm silicates, 50\% carbonaceous grains &  6, 7\\
    \vspace{0.1cm}
    Zubko-Crich-aringer & 2,000 & C & COMARCS & Amorphous carbon grains &  1, 7\\
    \vspace{0.1cm}
    Zubko-Crich-bb & 2,000 & C & BB & Amorphous carbon grains &  7\\
    \vspace{0.1cm}
    H11-LMC & 90,899 & C & COMARCS & Dust-growth grid with 1/2 solar metallicity &  5\\
    \vspace{0.1cm}
    H11-SMC & 91,058 & C & COMARCS & Dust-growth grid with 1/5 solar metallicity &  5\\
    \vspace{0.1cm}
    J1000-LMC & 85,392 & C & COMARCS & Dust-growth grid with 1/2 solar metallicity &  5\\
    \vspace{0.1cm}
    J1000-SMC & 85,546 & C & COMARCS & Dust-growth grid with 1/5 solar metallicity & 5\\
    \\
    \hline
    \vspace{-0.3cm}
    \end{tabular}
    \begin{flushleft}
    {{References}: \textbf{1}:\,Aringer et al. (2016), \textbf{2}:\,Begemann et al. (1997), \textbf{3}:\,Henning et al. (1995), \textbf{4}:\,Jaeger et al. (1998), \textbf{5}:\,Nanni et al. (2019), \textbf{6}:\,Ossenkopf et al. (1992), \textbf{7}:\,Zubko et al. (1996)}
    \end{flushleft}

\end{table*}

\thispagestyle{empty}


\end{document}
