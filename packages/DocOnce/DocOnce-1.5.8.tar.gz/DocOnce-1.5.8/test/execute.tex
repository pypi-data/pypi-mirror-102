%%
%% Automatically generated file from DocOnce source
%% (https://github.com/doconce/doconce/)
%%
%%
%-------------------- begin preamble ----------------------
\documentclass[%
oneside,                 % oneside: electronic viewing, twoside: printing
final,                   % draft: marks overfull hboxes, figures with paths
10pt]{article}
\listfiles               %  print all files needed to compile this document
\usepackage{relsize,makeidx,color,setspace,amsmath,amsfonts,amssymb}
\usepackage[table]{xcolor}
\usepackage{bm,ltablex,microtype}
\usepackage[pdftex]{graphicx}
\usepackage{fancyvrb,anslistings} % packages needed for verbatim environments
\usepackage{minted}
\usemintedstyle{default}
\usepackage[T1]{fontenc}
%\usepackage[latin1]{inputenc}
\usepackage{ucs}
\usepackage[utf8x]{inputenc}
\usepackage{lmodern}         % Latin Modern fonts derived from Computer Modern
% Hyperlinks in PDF:
\definecolor{linkcolor}{rgb}{0,0,0.4}
\usepackage{hyperref}
\hypersetup{
    breaklinks=true,
    colorlinks=true,
    linkcolor=linkcolor,
    urlcolor=linkcolor,
    citecolor=black,
    filecolor=black,
    %filecolor=blue,
    pdfmenubar=true,
    pdftoolbar=true,
    bookmarksdepth=3   % Uncomment (and tweak) for PDF bookmarks with more levels than the TOC
    }
%\hyperbaseurl{}   % hyperlinks are relative to this root
\setcounter{tocdepth}{2}  % levels in table of contents
% prevent orhpans and widows
\clubpenalty = 10000
\widowpenalty = 10000
% --- end of standard preamble for documents ---
% insert custom LaTeX commands...
\raggedbottom
\makeindex
\usepackage[totoc]{idxlayout}   % for index in the toc
\usepackage[nottoc]{tocbibind}  % for references/bibliography in the toc
%-------------------- end preamble ----------------------
\begin{document}
% matching end for #ifdef PREAMBLE
\newcommand{\exercisesection}[1]{\subsection*{#1}}
\renewcommand{\u}{\pmb{u}}
\newcommand{\f}{\bm{f}}
\newcommand{\xbm}{\bm{x}}
\newcommand{\normalvecbm}{\bm{n}}
\newcommand{\ubm}{\bm{u}}

\newcommand{\x}{\pmb{x}}
\newcommand{\normalvec}{\pmb{n}}
\newcommand{\Ddt}[1]{\frac{D#1}{dt}}
\newcommand{\halfi}{1/2}
\newcommand{\half}{\frac{1}{2}}
\newcommand{\report}{test report}
% ------------------- main content ----------------------
\begin{minted}[fontsize=\fontsize{9pt}{9pt},linenos=false,mathescape,baselinestretch=1.0,fontfamily=tt,xleftmargin=7mm]{fortran}
      subroutine test()
      integer i
      real*8 r
      r = 0
      do i = 1, i
         r = r + i
      end do
      return
C     END1

      program testme
      call test()
      return

\end{minted}

\begin{Verbatim}[numbers=none,fontsize=\fontsize{9pt}{9pt},baselinestretch=0.95]
  File "<ipython-input-2-5e4d3c40ba99>", line 1
    subroutine test()
                  ^
SyntaxError: invalid syntax
\end{Verbatim}
% ------------------- end of main content ---------------
\end{document}

