\section{File puma\_namelist}
\subsection{Namelist INP}
\begin{tabular}{|l|c|l|l|}                                  
\hline                                                        
{\bf Name      }&Def.&Type & Description \\               
\hline                                                        
{\bf column    }&   0 & int & 1: initialize PLASIM for column runs \\
{\bf kick      }&  1 & int & 0: no noise initialization ($p_s$ = const.) \\         
{\bf           }&     &     & 1: random white noise                \\
{\bf           }&     &     & 2: Equator symmetric random white noise  \\
{\bf           }&     &     & 3: mode (1,2) no random initialization  \\
{\bf mars      }&   0 & int & 1: initialize PLASIM for planet Mars \\
{\bf mpstep    }&  45 & int & minutes per step (lenghth of timestep) \\
{\bf nadv      }&   1 & int & 1: switches horizontal advection on \\
{\bf ncoeff    }&   0 & int & spectral coefficients to print in {\bf wrspam}\\
{\bf ndel(NLEV)}&all 2& int & order of hyperdiffusion for each level (2*h)\\
{\bf ndiag     }&  12 & int & output interval for diagnostics [timesteps] \\   
{\bf ndiagcf   }&   0 & int & 1: turn on cloud forcing diagnostic \\
{\bf ndiaggp   }&   0 & int & 1: process franks gp-diagnostic arrays \\
{\bf ndiaggp2d }&   0 & int & number of additional 2-d gp-diagnostic arrays \\
{\bf ndiaggp3d }&   0 & int & number of additional 3-d gp-diagnostic arrays  \\
{\bf ndiagsp   }&   0 & int & 1: process franks sp-diagnostic arrays \\
{\bf ndiagsp2d }&   0 & int & number of additional 2-d sp-diagnostic arrays \\
{\bf ndiagsp3d }&   0 & int & number of additional 3-d sp-diagnostic arrays \\
{\bf ndl(NLEV) }&all 0& int & 1: activate spectral printouts for this level \\
{\bf neqsig    }&   1 & int & 1: use equidistant sigma levels \\
{\bf nflux     }&   1 & int & 1: switches vertical diffusion on \\
{\bf ngui      }&   0 & int & 1: run with active GUI \\
{\bf nhdiff    }&  15 & int & critical wavenumber for horizontal diffusion \\
{\bf nhordif   }&   1 & int & 1: switches horizontal diffusion on \\
{\bf nkits     }&   3 & int & number of short initial timesteps \\       
{\bf noutput   }&   1 & int & enables (1) or disables (0) output file \\
{\bf npackgp   }&   1 & int & 1: pack gridpoint fields on output \\
{\bf npacksp   }&   1 & int & 1: pack spectral fields on output \\
{\bf nperpetual}&   0 & int & radiation day for perpetual integration \\
{\bf nprhor    }&   0 & int & 1: grid point for print out (only for checks!) \\
{\bf nprint    }&   0 & int & 1: comprehensive print out (only for checks!) \\
{\bf nrad      }&   1 & int & 1: switches radiation on  \\
{\bf ntime     }&   0 & int & 1: turn on time use diagnostics \\
{\bf nwpd      }&   1 & int & number of writes per day (to puma\_output) \\
\hline
\end{tabular}

\newpage

Namelist INP continued \vspace{3mm} \\
\begin{tabular}{|l|c|l|l|}                                  
\hline                                                        
{\bf Name      }&Def.&Type & Description \\               
\hline                                                        
{\bf n\_days\_per\_month} & 30 & int & length of month for simple calendar \\
{\bf n\_days\_per\_year} & 360 & int & length of year for simple calendar or 365 \\
{\bf n\_run\_days} & -1 & int & Simulation time (days to run) \\
{\bf n\_run\_months} & 0 & int & Simulation time (months to run) \\
{\bf n\_run\_years} & 1 & int & Simulation time (years to run) \\
{\bf n\_start\_month} & 1 & int & Starting month \\
{\bf n\_start\_year} & 1 & int & Starting year \\
{\bf psurf     }&101100.0&real& global mean surface pressure [Pa] \\
{\bf restim(NLEV)}&all 15.0 & real  & restoration timescale for each level \\ 
{\bf sigh(NLEV)} & all 0.0 & real & user definable sigmah array \\
{\bf sellon}    & 0.0 & real & longitude of soundings in the GUI \\
{\bf t0(NLEV)  }&all 250.0&real& reference $T_R$-temperature profile \\ 
{\bf tfrc(NLEV)}& 0,0,0,.. ,1  & int & Rayleigh friction timescale $\tau_F$ in days \\
{\bf tdissd    }& 0.2 & real  & diffusion time scale for divergence [days] \\
{\bf tdissq    }& 5.6 & real  & diffusion time scale for specific humidity [days] \\
{\bf tdisst    }& 5.6 & real  & diffusion time scale for temperature [days] \\
{\bf tdissz    }& 1.1 & real  & diffusion time scale for vorticity [days] \\
{\bf time0  }& 0.0 & real  & start time (for performance estimates) \\
\hline                                                        
\end{tabular}


\subsection{Namelist PLANET}
\begin{tabular}{|l|c|l|l|}                                  
\hline                                                        
{\bf Name      }&Def.&Type & Description \\               
\hline                                                        
{\bf akap         } & 0.286 & real & kappa \\
{\bf alr          } & 0.0065 & real & lapse rate \\
{\bf eccen        } & 0.0  & real & eccentricity for fixed orbits \\
{\bf ga           } & 9.81 & real & gravity \\
{\bf gascon       } & 287.0 & real & gas constant \\
{\bf mvelp        } & 0.0  & real & longitude of vernal equinox for fixed orbits (deg) \\
{\bf nfixorb      } & 0   & int & 1: fix the planetary orbit \\
{\bf obliq        } & 0.0  & real & obliquity for fixed orbits (deg) \\
{\bf plarad       } & 6371000.0 & real & planetary radius \\
{\bf pnu          } & 0.1 & real & time filter \\
{\bf ra1          } & 610.78 & real & parameter in Magnus-Teten formula \\
{\bf ra2          } & 17.269 & real & parameter in Magnus-Teten formula \\
{\bf ra4          } & 35.86 & real & parameter in Magnus-Teten formula \\
{\bf solar\_day   } & 86400.0 & real & length of solar day \\
{\bf siderial\_day} & 86164.0 & real & length of siderial day \\
{\bf ww           } & 7.29e-5 & real & $ 2 \pi / siderial day $ \\
{\bf yplanet      } & "Earth" & char & name of planet \\
\hline                                                        
\end{tabular}


\subsection{Namelist MISCPAR}
\begin{tabular}{|l|c|l|l|}                                  
\hline                                                        
Name   & Def. & Type & Description \\               
\hline                                                        
{\bf nfixer} & 1 & int & 1: correct negative moisture \\
{\bf nudge } & 0 & int & 1: temperature relaxation in the uppermost level \\
{\bf tnudge} & 10.0 & real & Time scale [d] of the temperature relaxation \\
\hline                                                        
\end{tabular}


\subsection{Namelist FLUXPAR}
\begin{tabular}{|l|c|l|l|}                                  
\hline                                                        
Name   & Def. & Type & Description \\               
\hline                                                        
{\bf nevap  }& 1 & int & 1: turn on surface evaporation \\  
{\bf nshfl  }& 1 & int & 1: turn on surface sensible heat flux \\
{\bf nstress }& 1 & int & 1: turn on surface wind stress \\
{\bf nvdiff  }& 1 & int & 1: turn on vertical diffusion \\
{\bf vdiff\_lamm  }& 160.0 & real & tuning parameter for vert. diff. \\
{\bf vdiff\_b  }& 5.0 & real & tuning parameter for vert. diff. \\
{\bf vdiff\_c }& 5.0 & real & tuning parameter for vert. diff. \\
{\bf vdiff\_d  }& 5.0 & real & tuning parameter for vert. diff. \\
{\bf zumin }& 1.0 & real & minimum surface wind speed (m/s) \\
\hline                                                        
\end{tabular}

  
\subsection{Namelist RADPAR}
\begin{tabular}{|l|c|l|l|}                                  
\hline                                                        
Name   & Def. & Type & Description \\               
\hline                                                        
{\bf acl2(3) } & 0.05,0.1,0.2 & real & cloud absorptivities spectral range 2 \\
{\bf acllwr  } & 0.1 & real & mass absorption coefficient for clouds (lwr) \\
{\bf clgray  } & -1.0 & real & cloud grayness \\
{\bf co2     } & 360.0 & real & co2 concentration (ppmv) \\
{\bf dawn    } & 0.0 & real & zenith angle threshhold for night \\
{\bf gsol0   } & 1365.0 & real & solar constant (w/m2) \\
{\bf iyrbp   } & -50 & int & Year before present (1950 AD); default = 2000 AD \\
{\bf ndcycle } & 0 & int & switch for daily cycle 1=on/0=off \\
{\bf nlwr    } & 1 & int & switch for long wave radiation (dbug) 1=on/0=off \\
{\bf no3     } & 1 & int & switch for ozon 1=on/0=off \\
{\bf nrscat  } & 1 & int & switch for rayleigh scattering (dbug) 1=on/0=off \\
{\bf nsol    } & 1 & int & switch for solar insolation (dbug) 1=on/0=off \\
{\bf nswr    } & 1 & int & switch for short wave radiation (dbug) 1=on/0=off \\
{\bf nswrcl  } & 1 & int & switch for computed or prescribed cloud props. 1=com/0=pres \\
{\bf rcl1(3) } & 0.15,0.3,0.6 & real & cloud albedos spectral range 1 \\
{\bf rcl2(3) } & 0.15,0.3,0.6 & real & cloud albedos spectral range 2 \\
{\bf th2oc   } & 0.04 & real & absorption coefficient for h2o continuum \\
{\bf tpofmt  } & 1.0 & real & tuning of point of mean (lwr) transmissivity in layer \\
{\bf tswr1   } & 0.04 & real & tuning of cloud albedo range1 \\
{\bf tswr2   } & 0.048 & real & tuning of cloud back scattering c. range2 \\
{\bf tswr3   } & 0.004 & real & tuning of cloud s. scattering alb. range2 \\
\hline                                                        
\end{tabular}


\subsection{Namelist RAINPAR}
\begin{tabular}{|l|c|l|l|}                                  
\hline                                                        
Name   & Def. & Type & Description \\               
\hline                                                        
{\bf clwcrit1   } & -0.1 & real & 1st critical vertical velocity for clouds \\
{\bf clwcrit2   } &  0.0 & real & 2nd critical vertical velocity for clouds \\
{\bf kbetta     } &  1   & int  & switch for betta in kuo (1/0=yes/no) \\
{\bf ncsurf     } &  1   & int  & conv. starts from surface (1/0=yes/no) \\
{\bf ndca       } &  1   & int  & dry convective adjustment (1/0=yes/no) \\
{\bf nmoment    } &  0   & int  & momentum mixing (1/0=yes/no) \\
{\bf nshallow   } &  0   & int  & switch for shallow convection (1/0=yes/no) \\
{\bf nprc       } &  1   & int  & large convective precip (1/0=yes/no) \\
{\bf nprl       } &  1   & int  & switch for large scale precip (1/0=yes/no) \\
{\bf rcrit(NLEV)} &      & real & critical relative hum. for non conv. clouds \\
\hline                                                        
\end{tabular}

  
\subsection{Namelist SURFPAR}
\begin{tabular}{|l|c|l|l|}                                  
\hline                                                        
Name   & Def. & Type & Description \\               
\hline                                                        
{\bf noromax  }& model resolution (NTRU) & int & resolution of orography \\
{\bf nsurf  }& not active & int & debug switch \\  
{\bf oroscale  }& 1.0 & real & scaling factor for orography \\
\hline                                                        
\end{tabular}


\section{File land\_namelist}
\subsection{Namelist LANDPAR}
\begin{tabular}{|l|c|l|l|}                                  
\hline                                                        
Name   & Def. & Type & Description \\               
\hline                                                        
{\bf  albgmax  } & 0.8   & real & max. albedo for glaciers \\
{\bf  albgmin  } & 0.6   & real & min. albedo for glaciers \\
{\bf  albland  } & 0.2   & real & albedo for land \\
{\bf  albsmax  } & 0.8   & real & max. albedo for snow \\
{\bf  albsmaxf } & 0.4   & real & max. albedo for snow (with forest) \\
{\bf  albsmin  } & 0.4   & real & min. albedo for snow \\
{\bf  albsminf } & 0.3   & real & min. albedo for snow (with forest) \\
{\bf  co2conv  } & 14.0  & real & co2 conversion factor \\
{\bf  drhsfull } & 0.4   & real & threshold above which drhs=1 [frac. of wsmax] \\
{\bf  drhsland } & 0.25  & real & wetness factor land \\
{\bf  dsmax    } & 5.00  & real & maximum snow depth (m-h20; -1 = no limit) \\
{\bf  dsoilz(NLSOIL)} &       & real & soil layer thickness \\
{\bf  dwatcini}  & 0.0   & real & soil water content (m) for manual 
                                  initialization   \\
                 &       &      &      (nwatcini=1) \\
{\bf  dz0land  } & 2.0   & real & roughness length land \\
{\bf  dzglac   } & -1.   & real & threshold of orography to be glacier (-1=none) \\
{\bf  dztop    } & 0.20  & real & thickness of the uppermost soil layer (m) \\
{\bf  forgrow  } & 1.0   & real & growth factor initialization \\
{\bf  gs       } & 1.0   & real & stomatal conductance initialization \\
{\bf  nbiome   } & 0     &  int & switch for vegetation model (1/0 : prog./clim) \\
{\bf  ncveg    } & 1     &  int & compute new dcveg (0=keep initial state) \\
{\bf  newsurf  } & 0     &  int & (dtcl,dwcl) 1: update from file, 2:reset  \\
{\bf  nlandt   } & 1     &  int & switch for land model (1/0 : prog./clim) \\
{\bf  nlandw   } & 1     &  int & switch for soil model (1/0 : prog./clim) \\
{\bf  nwatcini } & 0     &  int & switch for manual soil water setting (1/0 : on/off) \\
{\bf  rinisoil } &  0.0  & real & soil carbon initialization  \\
{\bf  riniveg  } &  0.0  & real & biomass carbon initialization \\
{\bf  rlaigrow } & 0.5   & real & above ground growth factor initialization \\
{\bf  rlue     } &  8.0E-10 & real & \\
{\bf  rnbiocats} &  0.0     & real & \\
{\bf  tau\_soil } & 42.0  & real & [years] - in landini scaled to seconds \\
{\bf  tau\_veg  } & 10.0  & real & [years] - in landini scaled to seconds \\
{\bf  wsmax    } & WSMAX\_EARTH & real & max field capacity of soil water (m) \\
{\bf  z0\_max   } & 2.0  &  real & maximum roughness length for vegetation \\
\hline                                                        
\end{tabular}


\section{File sea\_namelist}
\subsection{Namelist SEAPAR}
\begin{tabular}{|l|c|l|l|}                                  
\hline                                                        
Name   & Def. & Type & Description \\               
\hline                                                        
{\bf ncpl\_atmos\_ice  }& 32 & int & atmosphere ice coupling time steps \\
{\bf albsea  }& 0.069 & real & albedo for open water \\
{\bf albice  }& 0.7 & real & max. albedo for sea ice \\
{\bf dz0sea  }& $1.5\cdot 10^{-5}$ & real & roughness length sea [m]\\
{\bf dz0ice  }& $1.0\cdot 10^{-3}$ & real & roughness length ice [m]\\
{\bf drhssea  }& 1.0 & real & wetness factor sea \\
{\bf drhsice  }& 1.0 & real & wetness factor ice \\
\hline                                                        
\end{tabular}


\section{File ocean\_namelist}
\subsection{Namelist OCEANPAR}
\begin{tabular}{|l|c|l|l|}                                  
\hline                                                        
Name   & Def. & Type & Description \\               
\hline                                                        
{\bf dlayer(NLEV\_OCE)} & 50.0 & real & layer depth (m) \\
{\bf ndiag} & 480 & int & diagnostics each ndiag timestep \\
{\bf newsurf}& 0 & int & 1: read surface data after restart \\
{\bf nfluko}&  0 & int & switch for flux correction \\
{\bf nocean  }& 1 & int & ocean model (1) or climatology (0) \\  
{\bf nperpetual\_ocean}& 0 & int & perpetual climate conditions (day) \\
{\bf nprhor}& 0 & int & gridpoint for debug printout \\
{\bf nprint}& 0 & int & switch for debug printout \\
{\bf taunc}& 0.0 & real & time scale for newtonian cooling \\
{\bf vdiffk}& 1.0e-4 & real & vertikal diffusion coeff. [m**2/s] \\
\hline                                                        
\end{tabular}


\section{File ice\_namelist}
\subsection{Namelist ICEPAR}
\begin{tabular}{|l|c|l|l|}                                  
\hline                                                        
Name   & Def. & Type & Description \\               
\hline                                                        
{\bf newsurf}& 0 & int & 1: read surface data after restart \\
{\bf nfluko}&  0 & int & switch for flux correction \\
{\bf nice  }& 1 & int & sea ice model (1) or climatology (0) \\
{\bf nout  }& 32 & int & model data output every {\bf nout} time steps \\
{\bf nperpetual\_ice}& 0 & int & perpetual climate conditions (day) \\
{\bf nprhor}& 0 & int & gridpoint for debug printout \\
{\bf nprint}& 0 & int & switch for debug printout \\
{\bf nsnow}& 1 & int & allow snow on ice yes/no (1/0) \\
{\bf ntskin}& 1 & int & compute skin temperature (0=clim. \\
{\bf ncpl\_ice\_ocean  }& 1 & int & ice ocean coupling time steps \\
{\bf taunc}& 0.0 & real & time scale for newtonian cooling \\
{\bf xmind}& 0.1 & real & minimal ice thickness (m) \\
\hline                                                        
\end{tabular}


